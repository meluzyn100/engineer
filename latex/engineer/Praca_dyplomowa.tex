%%%%%%%%%%%%%%%%%%%%%%%%%%%%%%%%%%%%%%%%%%%%%%%%%%%%%%%%%
% Niniejszy plik przedstawia przykładowy skład 
% pracy dyplomowej na Wydziale Matematyki PWr. 
% 
% Autorzy: 
% Damian Fafuła
% Michał Kijaczko
% Jakub Michalczak
% Maciej Miśta
% Dagmara Nowak
% Tomasz Skalski
% Wojciech Słomian
%
%% Data utworzenia: 8.05.2018
% Numer wersji: 1
%
% Poniższą formatkę można rozpowszechniać i edytować 
% pod warunkiem zachowania numeru wersji, 
% informacji o autorach i dodaniu informacji 
% o wprowadzonych zmianach.
%
%%%%%%%%%%%%%%%%%%%%%%%%%%%%%%%%%%%%%%%%%%%%%%%%%%%%%%%%%
% Domyślną opcją jest: praca magisterska, język polski.
% W przypadku pracy pisanej w języku angielskim dodajemy 
% opcję [english].
% Dla pracy licencjackiej dodajemy opcję [licencjacka].
% Dla pracy inżynierskiej dodajemy opcję [inzynierska].
% Dopuszczalne są podwójne opcje, np. [licencjacka, english].
% Opcje dodajemy w kwadratowym nawiasie przy \documentclass.
%
%
%%%%%%%%%%%%%%%%%%%%%%%%%%%%%%%%%%%%%%%%%%%%%%%%%%%%%%%%%
\documentclass[inzynierska]{pwr_wmat_praca_dyplomowa}
%%%%%%%%%%%%%%%%%%%%%%%%%%%%%%%%%%%%%%%%%%%%%%%%%%%%%%%%%
%              DANE DO PRACY
%
% W przypadku pracy dyplomowej w języku angielskim nie jest konieczne 
% wypełnianie pól: \tytul{}, \kierunek{}, \specjalnosc{}, 
%                  \streszczenie{}, \slowakluczowe{}.
%%%%%%%%%%%%%%%%%%%%%%%%%%%%%%%%%%%%%%%%%%%%%%%%%%%%%%%%%
%
% Imię i nazwisko autora
\autor{Aleksander Jakóbczyk}
%
% Tytuł pracy dyplomowej 
\tytul{Zastosowanie stochastycznej
	optymalizacji do gier
	częściowo obserwowalnych} 
\tytulang{Tytuł pracy dyplomowej w języku angielskim}
%
% Tytuł / stopień / imię i nazwisko opiekuna
\opiekun{dr inż. Andrzej Giniewicz}
%
% Kierunek studiów wybieramy spośród następujących:
% 1) Matematyka
% 2) Matematyka i Statystyka
% 3) Matematyka stosowana
\kierunekstudiow{Matematyka stosowana}
%
% Kierunek studiów po angielsku wybieramy spośród następujących:
% 1) Mathematics
% 2) Mathematics and Statistics
% 3) Applied Mathematics
\kierunekstudiowang{Mathematics}
%
% Specjalność wybieramy spośród następujących: 
% KIERUNEK: Matematyka
% 1) Matematyka teoretyczna,
% 2) Statystyka matematyczna,
% 3) Matematyka finansowa i ubezpieczeniowa,
%
% KIERUNEK: Matematyka i Statystyka
% 4) Matematyka,
% 5) Statystyka i analiza danych, 
%
% 6) -- (w przypadku braku specjalizacji).
\specjalnosc{--} 
%
% Specjalność w języku angielskim wybieramy spośród następujących:
% KIERUNEK: Matematyka
% 1) Theoretical Mathematics,
% 2) Mathematical Statistics,
% 3) Financial and Actuarial Mathematics,
%
% KIERUNEK: Matematyka i Statystyka
% 4) Mathematics,
% 5) Statistics and Data Analysis,
%
% KIERUNEK: Applied Mathematics
% 6) Financial and Actuarial Mathematics, 
% 7) Mathematics for Industry and Commerce,
% 8) Computational Mathematics,
% 9) Modelling, Simulation and Optimization.
%
% 10) -- (w przypadku braku specjalizacji).
\specjalnoscang{Theoretical Mathematics} 
%
% Krótkie streszczenia po polsku i angielsku
% - nie dłuższe niż 530 znaków.
\streszczenie{Celem rozprawy jest wyznaczenie nieoczekiwanych strategii w grach częściowo obserwowalnych za pomocą metod
	stochastycznej optymalizacji. Praca będzie opierać się na wynikach Cauwet i Teytauda z 2018 roku, w których
	przedstawili nieoczekiwane strategie dla kilku klasycznych gier oraz kilku metod optymalizacji. Podjęta zostanie
	próba odtworzenia oraz rozszerzenia wyników na kolejną grę. Przeprowadzona zostanie analiza porównawcza dla
	różnych metod optymalizacji.}
\streszczenieang{Tutaj piszemy krótkie streszczenie pracy w języku angielskim (nie powinno być dłuższe niż 530 znaków).}
%
% Podajemy najważniejsze słowa kluczowe po polsku i angielsku
% - w obu przypadkach, nie więcej niż 150 znaków.
\slowakluczowe{tutaj podajemy najważniejsze słowa kluczowe (łącznie nie powinny być dłuższe niż 150 znaków).}  
\slowakluczoweang{tutaj podajemy najważniejsze słowa kluczowe w języku angielskim (łącznie nie powinny być dłuższe niż 150 znaków)}
%
%
%%%%%%%%%%%%%%%%%%%%%%%%%%%%%%%%%%%%%%%%%%%%%%%%%%%%%%%%%
% Definicje, lematy, twierdzenia, przykłady i wnioski
% Komendy wywołujące twierdzenia, definicje, itd., 
% czyli 'theorem', 'definition', 'corollary', itd., 
% można zmienić wedle uznania.
\theoremstyle{plain}
\newtheorem{theorem}{Twierdzenie}
\numberwithin{theorem}{chapter}
\newtheorem{lemma}[theorem]{Lemat} 
\newtheorem{corollary}[theorem]{Wniosek}
\newtheorem{fact}[theorem]{Fakt}
\theoremstyle{definition}
\numberwithin{theorem}{chapter}
\newtheorem{definition}[theorem]{Definicja} 
\newtheorem{example}[theorem]{Przykład}
\newtheorem{note}[theorem]{Uwaga}
%%%%%%%%%%%%%%%%%%%%%%%%%%%%%%%%%%%%%%%%%%%%%%%%%%%%%%%%%

\usepackage{amsmath}
\usepackage{amsthm}
\usepackage{amsfonts}
\usepackage{amssymb}
\usepackage{graphicx}
\usepackage{caption}
\usepackage{xcolor}
\usepackage{algpseudocode,algorithm,algorithmicx}
\usepackage{enumitem}

\DeclareRobustCommand{\bbone}{\text{\usefont{U}{bbold}{m}{n}1}}
\DeclareMathOperator{\EX}{\mathbb{E}}% expected value

%%%%%%%%%%%%%%%%%%%%%%%%%%%%%%%%%%%%%%%%%%%%%%%%%%%%%%%%%
%%%%%%%%%%%%%%%%%%%%%%%%%%%%%%%%%%%%%%%%%%%%%%%%%%%%%%%%%
\begin{document}
\frontmatter
\maketitle
\mainmatter
\tableofcontents
%\listoffigures
%\listoftables

{\backmatter \chapter{Wstęp}}
We wstępie zapowiadamy, o czym będzie praca. Próbujemy zachęcić czytelnika do dalszej lektury, np. krótko informując, dlaczego wybraliśmy właśnie ten temat i co nas w nim zainteresowało.

\chapter{Rozdział pierwszy}
Tabela \ref{tab:przykladowa} przedstawia przykładową tabelę. Do tworzenia tabeli służą m.in. środowiska \texttt{tabular} oraz \texttt{table}. Istnieje możliwość numeracji dwustopniowej, gdzie pierwsza cyfra oznacza numer rozdziału, a druga – kolejny numer tabeli w tym rozdziale. Tytuł powinien znajdować się centralnie nad tabelą, $12$ pkt odstępu od tekstu zasadniczego nad i pod tabelą wraz z tytułem. Jeśli tabela jest cytowana – należy podać centralnie pod tabelą źródło jej pochodzenia, np. opracowanie własne, opracowano na podstawie danych z GUS.
\begin{table}[ht]
\caption{Podstawowa Tabela}
\centering
\begin{tabular}{ccc}
\hline
\hline                       
Państwo & PKB (w milionach USD )& Stopa bezrobocia  \\  [0.5ex] 
\hline 
Stany Zjednoczone & 75 278 049 & 4,60\%  \\
Chiny & 11 218 281 & 4,10\%   \\
Japonia & 4 938 644 & 3,10\%  \\
Niemcy & 3 466 639 & 6,00\%   \\
Wielka Brytania & 2 629 188 & 4,60\%  \\ [1ex]  
\hline 
\end{tabular}
\caption*{\textit{Źródło: opracowanie własne}}
\label{tab:przykladowa} 
\end{table}

Do cytowania używamy komendy \texttt{cite}. W nawiasie klamrowym podajemy klucz, którego użyliśmy w pliku \emph{bibliografia.bib}. Przykład: \cite{einstein} lub \cite[chap. 2]{latexcompanion}.

\section{Podrozdział pierwszy}

\begin{table}[H]
\caption{Podstawowa Tabela}
\centering
\begin{tabular}{ccc}
\hline
\hline                       
Państwo & PKB (w milionach USD )& Stopa bezrobocia  \\  [0.5ex] 
\hline 
Stany Zjednoczone & 75 278 049 & 4,60\%  \\
Chiny & 11 218 281 & 4,10\%   \\
Japonia & 4 938 644 & 3,10\%  \\
Niemcy & 3 466 639 & 6,00\%   \\
Wielka Brytania & 2 629 188 & 4,60\%  \\ [1ex]  
\hline 
\end{tabular}
\caption*{\textit{Źródło: opracowanie własne}}
\label{tab:przykladowa2} 
\end{table}

\section{Podrozdział drugi}

Rysunki do pracy dyplomowej należy wstawiać w sposób podobny do wstawiania tabel, z~zasadniczą różnicą polegającą na tym, że podpis powinno umieszczać się centralnie pod rysunkiem, a nie powyżej niego. Numeracja i sposób cytowania pozostają bez zmian, przy czym tabele i rysunki nie mają numeracji wspólnej, np. po Tabeli \ref{tab:przykladowa2} występuje Rysunek \ref{rys1} (o ile jest to pierwszy rysunek rozdziału pierwszego), a nie Rysunek $1.3$.

\begin{figure}[ht]

\centering
                     
\includegraphics[scale=0.27]{logo_w13.jpg}
\caption{Podstawowy Rysunek}\label{rys1}
\end{figure}
\label{rys:przykladowy} 


\chapter{Definicje, lematy, twierdzenia, przykłady i wnioski}
Celem algorytmów których będziemy wykorzystywać w poniższej pracy jest znalezienie optymalnej strategi w grach częściowo obserwowalnych. Zdefiniujmy zatem podstawowe pojęcia potrzebne nam do tego aby matematycznie opisać czym jest gra i czym jest strategia optymalna.
W tym celu wprowadzi kilka podstawowych definicji.

\section{Typy gier }
Podstawowa kategorią na jaką możemy podzielić gry, jest podział ze względu na czas w którym gracze podejmują decyzje:

\begin{definition}[Gra w postaci strategicznej]
Jest to typ gry, w której gracze podejmują decyzje w tym samym
momencie.
\end{definition}
\begin{definition}[Gra w postaci ekstensywnej]
	Jest to typ gry, w której gracze podejmują decyzje we wcześniej
	ustalonej kolejności.
\end{definition}
Przykładami gier w postaci strategicznej sa gry papier-kamień-nożyce, oszust czy też mora. Natomiast przykładami gier w postaci ekstensywnej są szachy, warcaby oraz go.

Gry możemy również dzielić ze względu na posiadana wiedzę.
\begin{definition}[Gra z kompletną informacją]
	Jest to typ gry, w której gracze posiadają informacje o możliwych
	przyszłych wynikach gry i o zbiorach możliwych strategii.
\end{definition}
\begin{definition}[Gra częściowo obserwowalna]
	Jest to przeciwnie gier z kompletną informacją.
\end{definition}
Przykładami gier z kompletną informacją są szachy, warcaby oraz go.
Natomiast przykładami gier częściowo obserwowanymi są wszelkie gry posiadające w rozgrywce pewien elementy losowe takie jak rzut kostką czy też dobieranie kart. 

Istnieje jeszcze wiele innych podziałów gier ze wglądu na kategorie takie jak liczba graczy, ze względu na zbiory dostępnych akcji, ze względu na możliwość tworzenia koalicji i wiele wiele innych.

\section{Definicje i Oznaczenia}
Wprowadźmy podstawowe oznaczenia potrzebne nam do tego aby móc zdefiniować czym strategia optymalna.
\subsection{Strategie proste}
\begin{itemize}
	\item $ N = \{1,2,\dots, n\} $ -- zbiór graczy,
	\item $A_i,\; i \in N $ -- niepusty zbiór strategii  czystych gracza $i$,
	\item $m_i = |A_i|$ -- liczba strategi gracza $i$,
	\item $A = \displaystyle\prod_{i \in N} A_i$ -- zbiór wszystkich strategii gry, 
	\item $u_i : A \rightarrow \mathbb{R} $ -- funkcja wypłaty gracza $i$,
	\item $a=(a_1,a_2,\dots,a_n)=(a_i)_{i \in N},\; a_i \in A_i$ -- profil gry w strategiach czystych,
	\item $u_i(a) = u_i(a,a_{-i})$ -- wypłata gracza $i$ z profilu $a$,
	\item $a_{-i} = (a_i)_{i\in N \setminus \{i\}}$, -- profil wszystkich strategii poza strategią graca $i$.
\end{itemize}
	\begin{definition}[Gra strategiczna]
		Grą strategiczną nazywamy trójkę $GS = \langle N, (A_i)_{i \in N},(u_i)_{i \in N} \rangle $.
	\end{definition}
	
	\begin{definition}[Równowaga Nasha w strategiach czystych gry strategicznej]
		Równowaga Nasha w strategiach czystych gry strategicznej jest to taki profil gry $a^*= (a_1^*,a_2^*,\dots,a_N^*)\in A$, takim, że:
		\begin{align*}
			\displaystyle\mathop{\forall}_{i \in N} 
			\displaystyle\mathop{\forall}_{a_i \in A_i} \quad
			u_i(a_i^*,a_{-i}^*) \ge u_i(a_i, a_{-i}^*)
		\end{align*}
	\end{definition}
	Zatem jest to profil gry w którym istnieje strategia czystej dająca niegorszeń  wyniki od dowolnej innej strategii czystej.
	Okazuje się jednak że taki stan nie zawsze istnieje w strategiach czystych, np. w grze kamień-papier-nożyce strategia grania tylko kamienia daje gorsze rezultat przeciwko graniu tylko papieru. Podobnie z strategią grania tylko nożyc i grania tylko papieru.
	\subsection{Strategie mieszane}
	\begin{definition}[Strategia mieszana]
		Strategia mieszana $\sigma_i$ graca $i$ w grze strategicznej $GS = \langle N, (A_i)_{i \in N},(u_i)_{i \in N} \rangle $. Nazywamy rozkład prawdopodobieństwa na zbiorze strategi czystych $A_i$:
		\begin{align*}
			\sigma_i = (\sigma_{i1}, \sigma_{i2},\dots,\sigma_{im_i})
		\end{align*}
	gdzie $\sigma_{ik}$ oznacza prawdopodobieństwo, że gracz $i$ zagra strategie czysta $k\in A_i$.  
	\end{definition}
	Wprowadźmy dodatkowe oznaczenia:
	\begin{fact}
		Strategia czysta jest szczególnym przypadkiem strategi mieszanych w którym prawdopodobieństwo zagrania jednej z dostępnych strategii wynosi 1.
	\end{fact}
	Wprowadźmy dodatkowe oznaczenia:
 	\begin{itemize}
 		\item $\Sigma_i  = \left\{ \sigma_i: A_i \rightarrow [0,1],\displaystyle\sum_{k=1}^{n} \sigma_{ik} = 1, \sigma_{ki}\ge 0 \right\}$ -- zbiór strategii mieszanych gracza $i$,
 		\item  $\sigma = (\sigma_1, \sigma_2,\dots,\sigma_n)$ -- profil gry,
 		\item $u_i(\sigma) = u_i(\sigma_i,\sigma_{-i})$ -- wypłata gracza $i$ z profilu $\sigma$,
 		\item $\sigma_{-i} = (\sigma_i)_{i\in N \setminus \{i\}}$, -- profil wszystkich strategii poza strategią graca $i$. 
 	\end{itemize}
	\begin{definition}[Równowaga Nasha w strategiach mieszanej gry strategicznej]
		Profil gry strategicznej $\sigma_i^*$ jest Równowagą Nasha gdy:
		\begin{equation*}
			\displaystyle\mathop{\forall}_{i \in N} 
			\displaystyle\mathop{\forall}_{\sigma_i \in \Sigma_i} \quad
			u_i(\sigma_i^*,\sigma_{-i}^*) \ge u_i(\sigma_i, \sigma_{-i}^*)
		\end{equation*}
	Równowaga Nasha interpretujemy jako taki profil gry, w którym żaden z graczy nie opłaca się zmieniać swojej strategi, ponieważ nie skutkuje to zwiększeniem swoich zysków.
	\end{definition}

	\section{Twierdzenia}
	Algorytmy wykorzystywane w poniższej pracy oparte są o dwa twierdzenia a dokładniej o szczególne przypadki wynikające z nierówności \ref{Hoffding ineq} i \ref{Bernsteina emp ineq}:
	\begin{theorem}[Nierówność Hoffdinga]
		\label{Hoffding ineq}
		Niech $X_1,X_2,\dots,X_t$ będzie ciągiem niezależnych zmiennych losowych (i.i.d.) takim, że $a_i \le X_i \le b_i$, wtedy:
		\begin{gather*}
			S_t = \sum_{i=1}^{t} X_i,\quad c_i = b_i - a_i, \\
			P(|S_t - \EX(S_t)| \ge \epsilon ) \le 2\exp\left( -\frac{2\epsilon^2}{\sum_{i=1}^{n} c_i^2} \right)
		\end{gather*}
	\end{theorem}
	\begin{lemma}
		\label{Hoffding ineq lemma}
		Niech $X_1,X_2,\dots,X_t$ będzie ciągiem niezależnych zmiennych losowych (i.i.d.) takim, że $0 \le X_i \le 1$, wtedy:
		\begin{gather*}
			\overline{S}_t = \frac{S_t}{t},\quad 
			\mu = \EX(\overline{S}_t),\quad  	
			P(|\overline{S}_t - \mu | \le \epsilon ) = 1 - \delta, \\
			\epsilon \le  \sqrt{\frac{\ln(2/\delta)}{2t}}   
		\end{gather*}
	\end{lemma}

		\begin{theorem}[Empiryczna nierówność Bernsteina]
		\label{Bernsteina emp ineq}
		Niech $X_1,X_2,\dots,X_t$ będzie ciągiem niezależnych zmiennych losowych (i.i.d.) takim, że $a \le X_i \le b$, wtedy:
		\begin{gather*}
			P(|S_n - \EX(E_n)| \ge \epsilon ) = \delta, \quad \overline \sigma_t^2 = \frac{1}{t}\sum_{i=i}^{t}(S_i - \overline{S}_t)^2, \\
			|S_t - \EX(E_t)| \le \overline{\sigma}_t \sqrt{\frac{2\ln(3/\delta)}{t}} + \frac{3 R \ln{(3 / \delta)}}{t}
		\end{gather*}
	\end{theorem}
	\begin{lemma}\label{Bernsteina emp ineq lemma}
		Niech $X_1,X_2,\dots,X_t$ będzie ciągiem niezależnych zmiennych losowych (i.i.d.) takim, że $0 \le X_i \le 1$, wtedy:
		\begin{gather*}
			P(|\overline{S}_t - \mu | \le \epsilon ) = 1 - \delta,\\
			\epsilon \le \overline{\sigma}_t \sqrt{\frac{2\ln(3/\delta)}{t}} + \frac{3  \ln{(3 / \delta)}}{t}
		\end{gather*}
	\end{lemma}
	\section{Problem porównań wielokrotnych}
	Załóżmy, że z prawdopodobieństwem $1-\delta$ chcemy wiedzieć który z 2 graczy, $p_1$ i $p_2$ jest lepszy. W tym
	celu będziemy przeprowadzać testy statystyczne, dla których
	prawdopodobieństwo pomyłki k-tego testu wynosi $\delta_k$, aż do momentu, gdy jeden z graczy wygra przeważającą ilość razy. Wtedy po przeprowadzeniu $n$ takich testów
	\begin{gather*}
		P(\text{Chociaż jeden z $n$ testów się pomylił}) \overset{(*)}{\le} \sum_{k=1}^n \delta_k \implies  \\
	P(\text{Żaden test się nie pomylił}) \le 1 - \sum_{k=1}^n \delta_k \\
	\hspace*{-10cm} (*) \quad P(X+Y) \le P(X) + P(Y) 
	\end{gather*} 
	Oznacza to, że musimy wprowadzić pewną korektę, aby
	ostateczne prawdopodobieństwo popełnienia błędu było
	mniejsze niż $\delta$.
	Możemy wprowadzić jedna z dwóch poprawek:

	\begin{enumerate}[label=\thesection.\arabic*]
		\item \label{korekta 1} Niech $n$ będzie maksymalną liczbą testów jaką pozwalamy wykonać, aby wyznaczyć lepszego
		gracza. Wtedy $\delta_k=\frac{\delta}{n}$.
		\item \label{korekta 2} Niech $\delta_k$ spełnia nierówność $ \delta \ge \displaystyle\sum_{k = 1}^{\infty}\delta_k$. Wtedy niezależnie od
		ilości przeprowadzonych testów, ostateczne
		prawdopodobieństwo pomyłki będzie nie większe niż
		$\delta$.
	\end{enumerate}

	\section{Algorytmy wyścigowe}
	Jednymi z algorytmów stosującymi dane korekty są algorytmy wyścigowe
	(Racing \\Algorytms). Dwoma najpopularniejszymi typami algorytmów racingowych są "Hoeffding race" oraz "Bernstein race".
	Oparte są one odpowiednio o Twierdzenie \ref{Hoffding ineq} i Twierdzenie \ref{Bernsteina emp ineq lemma}. Mają one jednak pewną wadę, mogą zajmować one dużą ilość czasu, a w momencie gdy poziom umiejętności porównywanych graczy jest sobie równy (prawdopodobieństwo wygranej wynosi 50\%), wtedy z prawdopodobieństwem równym $1-\delta$ algorytm się nigdy nie zatrzyma.
	
	W celu poradzenia sobie z problemami, jakie wiążą się z klasycznymi algorytmami wyścigowymi, wprowadzimy  tzw. Limited Racing algorytm. Załóżmyzatem dodatkowy warunek, który mówi, że przerywamy działanie algorytmu w momencie, gdy empiryczna wartość oczekiwana z prawdopodobieństwem większym bądź równym $1-\delta$ jest znana z dokładnością co do zadanego $\epsilon$.
	\subsection{Bernstein race without maximum race length}
	Wykorzystując Lemat~\ref{Bernsteina emp ineq lemma} oraz korektę \ref{korekta 2} z $\delta_k=\frac{c\delta}{k^2}, c=\frac{6}{\pi^2}$ otrzymujemy, że dla ciągu $X_1,X_2,\dots,X_t$ i.i.d. takim, że  $0 \le X_i \le 1$:
	\begin{align*}
		\label{Bernstein race without maximum race length}
		\epsilon_{t,k} \le \overline{\sigma}_t \sqrt{\frac{2\ln(3/\delta_k)}{t}} + \frac{3  \ln{(3 / \delta_k)}}{t} =
		\overline{\sigma}_t\sqrt{\frac{2\ln(\frac{k^2\pi^2}{2\delta})}{t}} + \frac{3  \ln{(\frac{k^2\pi^2}{2\delta})}}{t}
	\end{align*}
	Wtedy $e_{t,k}$ interpretujemy jako maksymalną różnice miedzy empiryczna a teoretyczna wartością oczekiwaną po przeprowadzeniu $k$ testów i rozegraniu $t$ gier, z prawdopodobieństwem pomyłki równym $\delta$.
	\begin{fact}
		\label{Fakt Bernstein race without maximum race length}
		Przeprowadzać testy po każdej rozegranej grze ($k=t$)  $\lim\limits_{t\to\infty} e_{t,k} = 0$ 
	\end{fact}

	\begin{proof}[Dowód Faktu \ref{Fakt Bernstein race without maximum race length}]
		\begin{gather*}
		0\le X_i \le 1 \implies \overline{\sigma}_t^2 \le \frac{1}{2}\\
		\epsilon_{t,t} =\epsilon_t \le  \sqrt{\frac{\ln(\frac{t^2\pi^2}{2\delta})}{t}} + \frac{3  \ln{(\frac{t^2\pi^2}{2\delta})}}{t} \le \frac{4  \ln{(\frac{t^2\pi^2}{2\delta})}}{t}\\
		\forall_{a>0} \lim\limits_{t\to\infty} \frac{\ln(at^2)}{t} \overset{\left[\frac{\infty}{\infty}\right]}{=}\lim\limits_{t\to\infty} \frac{2}{t}=0 \implies
		\lim\limits_{t\to\infty} e_t \le 0 
		\end{gather*}
	Zatem $0\le \lim\limits_{t\to\infty} e_t \le 0 \implies \lim\limits_{t\to\infty} e_t = 0$
	\end{proof}
	
	
	
	
	
	
	
{\backmatter \chapter{Podsumowanie}}
Podsumowanie w pracach matematycznych nie jest obligatoryjne. Warto jednak na zakończenie krótko napisać, co udało nam się zrobić w pracy, a czasem także o tym, czego nie udało się zrobić.

{\backmatter \chapter{Dodatek}}
Dodatek w pracach matematycznych również nie jest wymagany. Można w nim przedstawić np. jakiś dłuższy dowód, który z pewnych przyczyn pominęliśmy we właściwej części pracy lub (np. w przypadku prac statystycznych) umieścić dane, które analizowaliśmy.

%%%%%%%%%%%%%%%%%%%%%%%%%%%%%%%%%%%%%%%%%%%%%%%%%%%%%%%%%
% BIBLIOGRAFIA
% W tworzeniu bibliografii najlepiej korzystać z BibTex'a, 
% który jest częścią systemu Tex. W naszym przypadku funkcję 
% przechowalni literatury, do której się odwołujemy, pełni 
% plik bibliografia.bib. Nie musimy ręcznie dodawać nowych 
% pozycji do bibliografii. Możemy wejść np. na stronę 
% https://mathscinet.ams.org/mathscinet/index.html, 
% znaleźć odpowiednią pozycję, wybrać ją, a następnie zmienić 
% 'Select alternative format' na BibTeX, skopiować uzyskany 
% tekst, wkleić do pliku bibliografia.bib i skompilować. 
% Gotowe informacje do pliku bibliografia.bib można znaleźć 
% także na https://arxiv.org - gdy znajdziemy interesującą nas 
% pracę, szukamy 'References & Citations' i klikamy 'NASA ADS', 
% a potem 'Bibtex entry for this abstract' 
% i postępujemy tak jak wcześniej.
%%%%%%%%%%%%%%%%%%%%%%%%%%%%%%%%%%%%%%%%%%%%%%%%%%%%%%%%%
\newpage
% w nawiasie klamrowym wpisujemy nazwę pliku z bibliografią w formacie .bib

\end{document}